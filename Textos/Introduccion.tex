%----------------------------------------------------------------------------------------
% Introducción
%----------------------------------------------------------------------------------------

Este manual contiene una colección de prácticas de programación para el curso de Inteligencia Artificial (IA), de la licenciatura en Ciencias de la Computación, de la Facultad de Ciencias de la UNAM.  Su objetivo es que sea utilizado en forma paralela a la impartición de los temas teóricos y no constituye un libro de texto teórico, aunque se presentan los antecedentes necesarios para realizar cada práctica.

Saber programar sistemas que utilicen técnicas de inteligencia artificial es una de las habilidades en el perfil profesional de nuestros estudiantes.  Para desarrollarla, hemos observado que la teoría no basta: es necesario escribir el código.  Aunque los estudiantes puedar responder perfectamente un examen sobre los algoritmos y conceptos teóricos, hay quienes se paralizan ante la perpectiva de tener que programarlos.  Por ello el curso incluye sesiones de laboratorio donde se cuenta con el apoyo de ayudantes que pueden dirigir ese proceso, de modo que, quienes aún sienten inseguridad, tengan a quien recurrir.  Para una correcta impartición de este laboratorio, se ha preparado este material en colaboración entre profesores y ayudantes, quienes han contribuido con propuestas de actividades, así como revisiones y mejoras a estas a lo largo de varios años.

Las prácticas están diseñadas para ofrecer la oportunidad de implementar los algoritmos y técnicas más significativos de la IA, aplicando los conocimientos adquiridos en materias previas como: Introducción a Ciencias de la Computación, donde se estudia programación orientada a objetos con \Java\ y se da una introducción a UML; Estructuras de datos, donde se implementan pilas, colas, listas, árboles y, en ocasiones, grafos; Análisis de algoritmos, donde se estudia el algoritmo de Dijkstra, entre otros, y Análisis lógico, donde se usa \code{Prolog} y se estudian el algoritmo de unificación y algoritmos de inferencia; y extendiéndo estos conocimientos con los elementos propios de las aplicaciones a IA.

El temario oficial del curso en que se basa este manual se puede consultar en la página de la facultad \hurl{https://www.fciencias.unam.mx/estudiar-en-ciencias/estudios/licenciaturas/ccomputacion}.  Dicho temario implica la división del estudio de los temas esenciales de la Inteligencia Artificial como sigue:

\begin{enumerate}
 \item Introducción: historia y agentes inteligentes.
 \item Inteligencia artificial clásica: resolución de problemas mediante búsquedas.
 \item Modelos probabilísticos.
 \item Aprendizaje automático o aprendizaje de máquina.
 \item Percepción y conocimiento.
\end{enumerate}

El curso está diseñado para ser una introducción general al campo de la IA.  El plan de estudios de la carrera incluye una serie de materias optativas donde es posible profundizar en cada tema, como serían Redes neuronales, Cómputo evolutivo, Razonamiento automatizado, Robótica móvil, Reconocimiento de patrones, y Reconocimiento de Patrones y Aprendizaje Automatizado.

En la práctica, dado que los estudiantes han llevado sus materias de matemáticas con una orientación más bien teórica: Álgebra lineal, Cálculo (en Matemáticas Aplicadas a las Ciencias) y Probabilidad, invertir el orden del aprendizaje automático y los modelos probabilistas ha resultado mucho más eficiente.  De este modo, se comienza con la IA clásica, que requiere la programación de sistemas simbólicos discretos, que resultan más familiares a los estudiantes, gracias a materias antecedentes como Estructuras de datos y Análisis de algoritmos.  Para esta parte, la bibliografía base en el libro de \cite{Russell2010}.  El orden de las prácticas obedece a dos criterios: el orden de exposición de los temas teóricos en dicho libro y el tiempo que tomará exponerlos en clase al tiempo que los estudiantes van programando las prácticas, de modo que se pueda realizar una práctica a la semana, dejando incluso dos semanas para las más laboriosas, sin que se dejen las prácticas antes de haber visto el tema en clase.

Posteriormente se hace un repaso de conceptos de cálculo para ingresar a los temas de aprendizaje automatizado, donde el cálculo y optimización de funciones son requisitos dominantes.  Para esta sección se usan libros como \cite{Nengnevitsky2005, Mitchell1997}, entre otros, que se van citando en cada práctica.

En tercer lugar trabajamos con el concepto de probabilidad y desarrollamos el razonamiento bajo incertidumbre haciendo uso de modelos gráficos probabilistas, tomando como base a \cite{KollerFriedman2009}, principalmente.  Para este momento del curso, los estudiantes ya se han acostumbrado al razonamiento abstracto, el manejo de grafos y la creación de modelos con interpretaciones en los ambientes de los agentes inteligentes.  Es en los proyectos propuestos donde se reúnen estos temas para dar origen a agentes capaces de percibir su entorno y actuar en él.

Si bien el manual trata de proponer actividades prácticas para los temas más importantes del semestre, nunca se han dejado todas las prácticas durante el curso, pues sería demasiado trabajo para ser completado a tiempo.   Cabe mencionar que, para las prácticas, no permitimos el trabajo en equipo, pues aún en el sexto semestre hemos identificado estudiantes que no han aprendido a programar bien y tienden a escudarse en colegas más hábiles en ello.  Con la finalidad de que desarrollen las habilidades con las que aún no se sientan cómodos, se les pide realizar estas prácticas de forma individual, contando con el apoyo del equipo académico cuando tienen dudas.

Por otro lado, de los proyectos propuestos, sólo se elige uno, que se comienza a desarrollar a la mitad del semestre y se entrega al final.  Para este proyecto final sí les permitimos trabajar en parejas, con el objetivo de también fomentar el trabajo en equipo, aunque también es posible desarrollarlo individualmente, para la evaluación ambos integrantes deben participar en la presentación del proyecto y deberán ser capaces de responder a las preguntas correspondientes.  Para apoyar al estudiantado, se proveé código auxiliar desde repositorios de \code{git}, que facilita la programación y/o introduce lenguajes y técnicas que pudieran resultarles novedosas.  Los lenguajes utilizandos son \Java, con las bibliotecas de \code{Processing} y \Python.

Finalmente, este material también puede ser aprovechado por el lector autodidacta que desee completar los ejercicios, pero requerirá de un buen libro de texto para estudiar los temas a profundidad.  Se incluyen varias sugerencias en la bibliografía del presente manual.

