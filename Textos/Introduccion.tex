%----------------------------------------------------------------------------------------
% Introducción
%----------------------------------------------------------------------------------------

Este manual contiene una colección de prácticas para el curso de Inteligencia Artificial, de la carrera en Ciencias de la Computación, de la Facultad de Ciencias de la UNAM.  El temario oficial se puede consultar en la página de la facultad \hurl{https://www.fciencias.unam.mx/estudiar-en-ciencias/estudios/licenciaturas/ccomputacion}.






Su objetivo principal es guiar a los estudiantes para implementar las estructuras básicas más utilizadas: \textit{pilas}, \textit{colas}, \textit{listas}, \textit{árboles de búsqueda}, \textit{diccionarios} y \textit{gráficas} considerando que otras estructuras más complejas se componen con los elementos utilizados en estas primeras.  Adicionalmente se incluye un ejercicio con algoritmos de ordenamiento, requerido por el programa del curso.

Aún cuando las estructuras de datos y su programación son estudiadas en detalle durante el curso, cada práctica incluye una sección de antecedentes con su definición y las características más importantes a manera de repaso y como referencia rápida para los estudiantes.   Además, cada práctica viene acompañada por código auxiliar para facilitar su programación: definiciones de interfaces a implementar, archivos \code{build.xml} para su compilación y distribución mediante la herramienta \code{ant} y un amplio sistema de pruebas unitarias diseñadas para otorgar retroalimentación mientras se desarrolla el código; algunas prácticas incluso vienen acompañadas por una interfaz gráfica en \code{JavaFX}, que permite visualizar las estructuras si los métodos están implementados correctamente.  Todos estos recursos se encuentran en repositorios plantilla de \code{git}, uno por cada práctica.  Los alumnos generarán sus propios repositorios a partir de estas plantillas.

Algunas de las prácticas requieren que los alumnos implementen interfaces diseñadas específicamente para este curso, pero la gran mayoría les requiere implementar las interfaces de la API de \Java.  Esto tiene el propósito de mantenerlos en contacto con interfaces de programación diseñadas profesionalmente y en uso en sistemas comerciales, así como tener un referente a la rigurosidad de la documentación que se debe entregar a los usuarios de estas interfaces, pretendiendo que adquieran ellos el mismo estándar.  Sin embargo, para reducir las oportunidades de plagiarismo, las clases que se implementan en estas prácticas tienen sus nombres en español.  Como un detalle curioso, varias de estas clases tienen elementos con acentos.  Esta práctica poco común tiene dos fines: que los alumnos comiencen a acercarse a los problemas inherentes de internacionalizar aplicaciones, viéndose en la necesidad de aplicar sus conocimientos sobre codificaciones de caracteres UTF-8 para evitar problemas durante la compilación y ejecución de su código; pero también apoyar la conservación de un buen manejo del idioma español pues, a la fecha, los trabajos de titulación llegan a manos de los sinodales adolesciendo de una cantidad impresionante de acentos en el texto.  Tal vez, si el compilador lo requiere, esas reglas de la gramática española se puedan ir afianzando un poco más en nuestros futuros profesionistas.

El uso de este manual durante ya varios semestres impartiendo la materia nos sugiere no utilizar todas las prácticas en un solo curso, pues la mayoría de los alumnos tiende a sentirse abrumado por las dificultades que le presenta implementar cada nueva técnica.  Por ello se recomienda, para cada estructura lineal, elegir sólo una de las implementaciones, ya sea con referencias o con arreglos.  Se presentan aquí ambas opciones por completez.  Las únicas restricciones ineludibles son: \emph{Colección abstracta} es requerida por todas la estructuras después de ella y \emph{Árbol binario ordenado} es requerida por \emph{Árbol AVL} y \emph{Árbol rojinegro}.  Igualmente, también hemos descartado el permitir equipos para realizar las tareas pues, desafortunadamente, hemos detectado que, con dicho sistema, hay alumnos que se las arreglan para llegar a los últimos semestres de la carrera tan sólo con habilidades de programación muy básicas, lo cual consideramos una profunda deficiencia para un profesionista de esta área.  Sin embargo, la decisión final está en manos del titular del curso, por lo que lo anterior son sólo las sugerencias que a nosotros nos han funcionado mejor, otro equipo pudiera obtener resultados diferentes.

Adicionalmente, la última sección del manual incluye sugerencias de proyectos donde se aplican las estructuras estudiadas a problemas diversos.  Algunos de estos proyectos se pueden asignar como ejercicios especiales durante el curso o también se han utilizado durante la aplicación de exámenes extraordinarios.  Finalmente, este material también puede ser aprovechado por el lector autodidacta que desee completar los ejericios, pero no sin la compañía de un buen libro de texto que exponga los temas en más detalle como los incluidos en la bibliografía del presente manual.

\section*{Sobre las actividades}
\addcontentsline{toc}{section}{Sobre las actividades}

Cada práctica establece metas y objetivos concretos.  Para alcanzarlos se proponen tres tipos de actividades.  La primeras están marcadas dentro de secciones etiquetadas precisamente como \emph{Actividad}.  Varias de ellas son para ser realizadas en apoyo al tema en estudio, pero no se pretende que sean evaluadas directamente, pues su impacto estará reflejado en la correcta solución de los ejercicios siguientes.  Algunas de estas actividades tienen una leyenda marcada con la etiqueta \textbf{Entregable:}, los productos de estas actividades se deben entregar al profesor ya sea en árchivos de código, dónde y cuándo así sea requerido, o en un reporte que acompañe la entrega de la práctica.  Dada la estructura de los códigos auxiliares en \code{git}, un formato adecuado es un reporte en formato \code{pdf} entregado en el directorio base del repositorio del alumno; otra alternativa es editar el archivo \code{README.md} para incluir ahí el reporte y someter los cambios.

En segundo lugar, están las actividades enumeradas en las secciones \emph{Desarrollo}.  Aquí se describen los requisitos al implementar las estructuras.  El producto esperado es el código correspondiente.

Finalmente, cada práctica concluye con una breve sección de preguntas, cuyas respuestas deben ser añadidas al reporte.
