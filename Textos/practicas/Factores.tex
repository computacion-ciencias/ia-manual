\section{Objetivo}

Que el alumno implemente una clase \classname{Factor} para familiarizarse con las operaciones entre factores que se utilizan para realizar inferencia en redes Bayesianas

\section{Introducción}

Un factor es una estructura matemática que nos ayudará a sistematizar las diversas operaciones requeridas para operar con distribuciones de probabilidad discretas.  Por ello, en esta práctica, el alumno se familiarizará con estas estructuras e implementará sus operaciones en un lenguaje de programación.  Posteriormente, la biblioteca creada será la herramienta con la cual se resolverán problemas de aplicación diversos.

\begin{definition}[Factor]
 \quotes{Sea \(D\) un conjunto de variables aleatorias [y $Val(D)$ una asignación de valores a estas variables]. Se define un Factor $\phi$ como una función de \(Val(D)\) a \(\mathbb{R}\). Un factor es no-negativo si todas sus entradas son no negativas. Al conjunto de variables \(D\) se le llama \textit{alcance del factor} y se denota como \(Alcance(\phi)\)}
 
 \parencite[104]{KollerFriedman2009}.
\end{definition}

Tomando como base la definición de Koller y Friedman, podemos realizar un parafraseo y afirmar que un Factor es \quotes{Una estructura matemática definida sobre un conjunto de variables donde cada variable puede tomar valores de su dominio, el cual debe ser finito. A este conjunto de variables se les denomina como el alcance del factor; el factor asocia un número real a cada posible asignación de valores de esas variables}.

En cuanto a las operaciones de Factores que se explicarán en esta práctica, nos basamos en las operaciones que presentan Koller y Friedman y que se utilizan en la materia: multiplicación, reducción, marginalización y normalización.

Utilizaremos a los factores para realizar operaciones con distribuciones de probabilidad, sin embargo es importante notar que, debido a la misma definición del Factor, el resultado de cada operación entre factores no siempre será una medida de probabilidad, pues esta limita su valores al intervalo $[0,1]$, mientras que los factores ocupan todo \(\mathbb{R}\).  Adicionalmente, sus operaciones no siempre incluyen las normalizaciones requeridas por las distribuciones de probabilidad; pero esto se convertirá en una ventaja para nosotros, como veremos más adelante.

\subsection[Operaciones]{Operaciones entre Factores}

\subsubsection{Multiplicación}

\begin{definition}[Multiplicación]
\quotes{Sean $X$, $Y$ y $Z$ tres conjuntos disjuntos de variables, y sean $\phi_1(X,Y)$ y $\phi_2(Y,Z)$ dos factores. Se define el factor producto   $\phi_1 \times \phi_2$ como un factor $\psi: Val(X,Y,Z) \mapsto \mathbb{R}$ de la forma siguiente: \[ \psi(X,Y,Z) = \phi_1(X,Y) \cdot \phi_2(Y,Z)\]}

\parencite[107]{KollerFriedman2009}.
\end{definition}

La operación de multiplicación es sencilla. Si los conjuntos de variables de los factores a multiplicar no tienen elementos en común, se multiplica cada entrada del factor A por cada entrada del Factor B. El alcance del factor resultante es la unión de los alcances de los factores a multiplicar \parencite[107]{KollerFriedman2009}.


\begin{figure}
    \centering
    \begin{subfigure}[b]{0.3\textwidth}
        \centering
        \begin{tabular}{ l | c }
          A & $\phi$(A)\\ \hline
          0 & .3  \\ \hline
          1 & .7  \\
        \end{tabular}
        \caption{Factor $A$}
    \end{subfigure}
    ~ 
    \begin{subfigure}[b]{0.3\textwidth}
        \centering
        \begin{tabular}{ l | c }
          B & $\phi$(B)\\ \hline
          0 & .6  \\ \hline
          1 & .4  \\
        \end{tabular}
        \caption{Factor $B$}
    \end{subfigure}
    ~
    \begin{subfigure}[b]{0.3\textwidth}
        \centering
        \begin{tabular}{ l | c }
          C & $\phi$(C)\\ \hline
          0 & .2  \\ \hline
          1 & .8  \\
        \end{tabular}
        \caption{Factor $C$}
    \end{subfigure}
    \caption{Factores sobre variables aleatorias binarias.}\label{fig:ABC}
\end{figure}

Por ejemplo: Tenemos los tres factores $A$, $B$ y $C$ de la \fref{fig:ABC}. Para obtener el factor $AB = A \times B$, se multiplicaría el renglón $A=0$ con $B=0$, luego $A=0$ con $B=1$, $A=1$ con $B=0$ y por último $A=1$ con $B=1$ como en la \fref{fig:FactorAB}.

\begin{figure}[H]
  \begin{center}
    \begin{tabular}{ l  c | c }
      A & B & $\phi$(A,B)\\ \hline
      0 & 0 & (.3)*(.6) = 0.18 \\ \hline
      0 & 1 & (.3)*(.4) = 0.12 \\ \hline
      1 & 0 & (.7)*(.6) = 0.42 \\ \hline
      1 & 1 & (.7)*(.4) = 0.28 \\
    \end{tabular}
  \end{center}
  \caption{Factor $AB$}
  \label{fig:FactorAB}
\end{figure}

\noindent Si el alcance de ambos factores tiene variables en común, se debe asegurar que tengan el mismo valor en cada renglón por multiplicar en ambos factores. Por ejemplo, si se tienen los factores $AB$ y $AC$, al momento de multiplicar el renglón $\{A=0,B=0\}$ se debe seleccionar los renglones donde $A=0$ en el factor $AC$, estos son $\{A=0,C=0\}$ y $\{A=0,C=1\}$, como muestran \fref{fig:FAFB} y \fref{fig:FactorABC}.

\begin{figure}[h!]
    \centering
    \begin{subfigure}[b]{0.4\textwidth}
        \centering
        \begin{tabular}{ l  c | c }
          A & B & $\phi$(A,B)\\ \hline
          0 & 0 & (.3)*(.6) = 0.18 \\ \hline
          0 & 1 & (.3)*(.4) = 0.12 \\ \hline
          1 & 0 & (.7)*(.6) = 0.42 \\ \hline
          1 & 1 & (.7)*(.4) = 0.28 \\
        \end{tabular}
        \caption{Factor $AB = A \times B$}
    \end{subfigure}
    ~ 
    \begin{subfigure}[b]{0.4\textwidth}
        \centering
        \begin{tabular}{ l  c | c }
          A & C & $\phi$(A,C)\\ \hline
          0 & 0 & (.3)*(.2) = 0.6 \\ \hline
          0 & 1 & (.3)*(.8) = 0.24 \\ \hline
          1 & 0 & (.7)*(.2) = 0.14 \\ \hline
          1 & 1 & (.7)*(.8) = 0.56 \\
        \end{tabular}
        \caption{Factor $AC = A \times C$}
    \end{subfigure}
    \caption{Multiplicandos.}
    \label{fig:FAFB}
\end{figure}


\begin{figure}[H]
  \begin{center}
    \begin{tabular}{ l  l  c | c }
      A & B & C & $\phi$(A,B,C)\\ \hline
      0 & 0 & 0 & [(.3)*(.6)]*[(.3)*(.2)] = 0.0108  \\ \hline
      0 & 0 & 1 & [(.3)*(.6)]*[(.3)*(.8)] = 0.0432 \\ \hline
      0 & 1 & 0 & [(.3)*(.4)]*[(.3)*(.2)] = 0.0072 \\ \hline
      0 & 1 & 1 & [(.3)*(.4)]*[(.3)*(.8)] = 0.0288 \\ \hline
      1 & 0 & 0 & [(.7)*(.6)]*[(.7)*(.2)] = 0.0588 \\ \hline
      1 & 0 & 1 & [(.7)*(.6)]*[(.7)*(.8)] = 0.2352 \\ \hline
      1 & 1 & 0 & [(.7)*(.4)]*[(.7)*(.2)] = 0.0392 \\ \hline
      1 & 1 & 1 & [(.7)*(.4)]*[(.7)*(.8)] = 0.1568 \\ 
    \end{tabular}
  \end{center}
  \caption{Factor $AB \times AC$}
  \label{fig:FactorABC}
\end{figure}

Es importante hacer notar que al multiplicar los factores $AB$ y $AC$ no se estaría obteniendo la probabilidad conjunta de $A$, $B$ y $C$, si no alguna otra función $\phi(A,B,C)$.

\subsubsection{Reducción}

\begin{definition}[Reducción]
\quotes{Sea $\phi(Y)$ un factor y \(U = u\) una asignación para \(U \subseteq Y\). Se define la reducción de un factor $\phi$ al contexto \(U = u\), denotado como \(\phi[U = u]\) (y abreviado como $\phi[u]$), como un factor con alcance \(Y' = Y - U\) de tal forma que \[\phi[u](y') = \phi(y',u)\]}

\parencite[111]{KollerFriedman2009}.
\end{definition}

La operación de reducción consiste en seleccionar un valor de alguna variable del factor y sólo tomar los renglones que cumplen con el valor dado de la variable. Por ejemplo: Se tiene el factor $AB$ de la \fref{fig:FactorAB}.

\begin{figure}[H]
  \begin{center}
    \begin{tabular}{ l  c | c }
      A & B & $\phi$(A,B)\\ \hline
      0 & 0 & .18  \\ \hline
      0 & 1 & .12  \\ \hline
      1 & 0 & .42  \\ \hline
      1 & 1 & .28  \\
    \end{tabular}
  \end{center}
  \caption{Factor $AB$}
  \label{fig:FactorAB}
\end{figure}

\noindent Si se desea reducir con $A = 0$, el resultado es el factor en la \fref{fig:FactorA0B}.

\begin{figure}[h]
  \begin{center}
    \begin{tabular}{ l | c }
      B & \(\phi(B)\)\\ \hline
      0 & .18  \\ \hline
      1 & .12  \\
    \end{tabular}
  \end{center}
  \caption{Factor \(B\)}
  \label{fig:FactorA0B}
\end{figure}

\subsubsection{Normalización}
Para normalizar un factor, esto es, que los valores que toma el factor sumen 1, basta con sumar todos los valores asociados a las asignaciones y dividir cada uno entre esta suma. Por ejemplo: Tenemos el factor \(\phi(B)\), la suma de sus posibles valores es .3, entonces el factor normalizado queda como en la \fref{fig:FactorA0BNorm}.

\begin{figure}[H]
  \begin{center}
    \begin{tabular}{ l  c | c }
      A & B &  \(\phi(B)\)\\ \hline
      0 & 0 & (.18/.3)= .6  \\ \hline
      0 & 1 & (.12/.3)= .4 \\
    \end{tabular}
  \end{center}
  \caption{Factor \(B\) normalizado}
  \label{fig:FactorA0BNorm}
\end{figure}

% \noindent Esta operación es útil cuando se desea calcular probabilidades condicionales.

\subsubsection{Marginalización}
\quotes{Sea $X$ un conjunto de variables y sea \(Y \notin X\) una variable y $\phi(X,Y)$ un factor. Se define la marginalización de $Y$ en $\phi$, denotado como \(\sum_Y \phi\), como un factor $\psi$ sobre $X$ de tal forma que \[\psi(X) = \sum_Y \phi(X,Y)\]} \parencite[297]{KollerFriedman2009}.

La operación de marginalización consiste en tomar la variable a marginalizar, sumar los valores en los renglones en que cambia su valor pero el de las demás variables no, y asignar esta suma al renglón correspondiente de las variables restantes. Por ejemplo: tenemos el factor AB y deseamos marginalizar la variable B. Entonces, tomamos los renglones donde A=0 y los sumamos, tomamos los renglones donde A=1 y los sumamos:

\begin{figure}[H]
    \centering
    \begin{subfigure}[b]{0.4\textwidth}
        \centering
        \begin{tabular}{ l l  c | c }
           & A & B & $\phi$(A,B)\\ \cline{2-4}
          \(\to\) & 0 & 0 & .18  \\ \cline{2-4}
          \(\to\) & 0 & 1 & .12  \\ \cline{2-4}
           & 1 & 0 & .42  \\ \cline{2-4}
           & 1 & 1 & .28  \\
        \end{tabular}
        \caption{Factor AB}
    \end{subfigure}
    ~ 
    \begin{subfigure}[b]{0.4\textwidth}
        \centering
        \begin{tabular}{l  l | c }
            & A & $\phi$(A)\\ \cline{2-3}
        \(\to\) & 0 & (.18)+(.12)  \\
        \end{tabular}
        \caption{Factor A}
    \end{subfigure}
    \caption{Paso 1}
\end{figure}

% \noindent y

\begin{figure}[H]
    \centering
    \begin{subfigure}[b]{0.4\textwidth}
        \centering
        \begin{tabular}{ l l  c | c }
           & A & B & $\phi$(A,B)\\ \cline{2-4}
           & 0 & 0 & .18  \\ \cline{2-4}
           & 0 & 1 & .12  \\ \cline{2-4}
          \(\to\) & 1 & 0 & .42  \\ \cline{2-4}
          \(\to\) & 1 & 1 & .28  \\
        \end{tabular}
        \caption{Factor AB}
    \end{subfigure}
    ~ 
    \begin{subfigure}[b]{0.4\textwidth}
        \centering
        \begin{tabular}{l  l | c }
            & A & $\phi$(A)\\ \cline{2-3}
            & 0 & (.18)+(.12)  \\ \cline{2-3}
        \(\to\) & 1 & (.42)+(.28)  \\
        \end{tabular}
        \caption{Factor A}
    \end{subfigure}
    \caption{Paso 2}
\end{figure}


\subsection{Desarrollo e implementaci\'on}

\noindent La práctica consiste en crear una clase \classname{Factor} que implemente la operaciones de multiplicación, reducción y normalización de factores y marginalización de variables. Todo esto utilizando el lenguaje de programación \classname{Python}.


\subsubsection{Implementaci\'on}

\begin{enumerate}
  \item Crear una clase \classname{Variable}, con los siguientes atributos: \classname{nombre} y \classname{valores\_posi \linebreak bles} y sobrescribir el método \classname{\_\_str\_\_}.
  % \item Programar un método \classname{tabla\_de\_valores} que reciba como parámetro una lista de objetos tipo \classname{Variable} y que regrese una matriz (lista de listas) que contenga todas las combinaciones de los valores que pueden tomar las variables. TIP: Inicializa la matriz con una única columna que contenga los valores posibles de la primer variable. Después, para cada variable restante, crear una nueva tabla. La nueva tabla es el resultado de anexar los valores posibles de la variable actual por cada renglón de la tabla anterior.
  \item Crear una clase \classname{Factor} que contenga los atributos: 
    \begin{itemize}
      \item \classname{alcance}: una lista de objetos de clase \classname{Variable}.
      \item \classname{valores}: una lista de valores asociados a cada renglón.
      % \item \classname{tabla\_valores}: el resultado del método del paso anterior.
    \end{itemize}
    
  \item Programar un método auxiliar \code{\_generar\_tabla\_de\_valores}, que sólo ejecutará su algoritmo la primera vez que sea invocado y su uso primordialmente será para poder imprimir en pantalla el contenido del factor, es decir, servirá para sobreescrbir el método \code{\_\_str\_\_} de la clase \code{Factor}.
  
  Este método deberá generar, de forma explícita, la tabla de asignaciones posibles a las variables en el alcance del factor.  Estas combinaciones serán listadas en formato semejante a un reloj digital, de modo que la última variable en el atributo \code{alcance} sea la que varíe su valor más rápido.  Los valores posibles se irán listando en el orden en que aparecen en el atributo \code{valores\_posibles} del objeto de tipo \code{Variable}.
  
  Observa que esta implementación no exige que los valores de la variables sean números, como indicaría la definición formal, pues en cualquier modo esto no es requisito para el correcto funcionamiento de las operaciones.  Esta generalización podrá ser utilizada para visualizar resultados más rápidos de interpretar.
  
  Por ejemplo, para las variables:
  \begin{itemize}
   \item \code{Letras = ['a', 'b']}
   \item \code{Útiles = ['cuaderno', 'lápiz', 'goma']}
   \item \code{Números = ['2', '8']}
  \end{itemize}
  La tabla de valores posibles se extiende como en la \fref{fig:tablavalores}.
  
  \begin{figure}
        \centering
        \begin{tabular}{r c c c}
         \tiny{Índice} & Letras & Útiles & Números \\ \hline
         \tiny{0} & a & cuaderno & 2 \\
         \tiny{1} & a & cuaderno & 8 \\
         \tiny{2} & a & lápiz    & 2 \\
         \tiny{3} & a & lápiz    & 8 \\
         \tiny{4} & a & goma     & 2 \\
         \tiny{5} & a & goma     & 8 \\
         \tiny{6} & b & cuaderno & 2 \\
         \tiny{7} & b & cuaderno & 8 \\
         \tiny{8} & b & lápiz    & 2 \\
         \tiny{9} & b & lápiz    & 8 \\
         \tiny{10} & b & goma     & 2 \\
         \tiny{11} & b & goma     & 8 \\
        \end{tabular}
        \caption{Tabla de valores}
        \label{fig:tablavalores}
  \end{figure}
  
  Usando esto, sobreescrbir el método \code{\_\_str\_\_}.

  \item Programar un método auxiliar que reciba como parámetro un diccionario, cuyas llaves sean variables y los valores sean el valor asignado a cada una.  Este método deberá obtener el índice en la tabla de valores que corresponda a esta asignación.  Observa que para obtener el índice no es necesario haber desarrollado la tabla explícitamente, se utiliza un polinomio de direccionamiento.
  
  Por ejemplo, el polinomio de direccionamiento para un factor de 3 variables (A, B y C) se vería así:
      \[\iacute ndice = (pos(a) * |B| * |C|) + (pos(b) * |C|) + pos(c)\]
    donde \(pos(a)\) es la posición del valor asociado a la variable \(A\) en su lista de valores posibles y \(|A|\) es el tamaño de esta lista.
    
  \textbf{TIP:} Pueden factorizar los términos comunes (los tamaños de las listas) e implementar una función auxiliar que calcule recursivamente el valor del polinomio de direccionamiento.
  
  \item Implementa las operaciones: multiplicación, reducción, normalización y marginalización.
  
  \begin{itemize}
   \item En el caso de la multiplicación, considera también el caso de multiplicar por un escalar.  Si el parámetro recibido es un escalar, el factor resultado tendrá el mismo alcance y sus valores serán multiplicados todos por el escalar.
   
   \item Para la marginalización, si se pide marginalizar la única variable del factor, suma los renglones y devuelve el escalar.
   
   \item En la reducción, si se reduce la única variable, devuelve el valor del factor que corresponda al valor indicado de la variable.
  \end{itemize}

  \textbf{TIP:} Crea primero el factor resultado. Para cada renglón en la tabla de valores de este factor, encuentra los renglones relevantes en los operandos utilizando el método auxiliar mencionado en el punto anterior y realiza la operación correspondiente.
\end{enumerate}

% \textbf{\\Entrada} (opcional)\bigskip
% 
% \noindent El programa deberá recibir la descripción de los factores en un archivo de texto. A continuación se define la sintaxis del archivo con los factores:
% 
% \begin{itemize}
%   \item Variables: 
%     \begin{align*}
%       \{&<Var_1>:[<val_0>, . . . , <val_m>],\\
%         &<Var_2>:[<val_0>, . . . , <val_m>], . . . ,\\
%         &<Var_n>:[<val_0>, . . . , <val_m>]\}
%     \end{align*}
%     Donde \(<Var_{i}>\) indica el nombre de la variable y \(<val_i>\) los valores que puede tomar la variable.
%     
%   \item Factores: 
%     \begin{align*}
%       [& [<Var_1>,<Var_2>, . . . , <Var_n>], \\
%        & . . . , \\
%        & [<Var_1>,<Var_2>, . . . , <Var_n>]]
%     \end{align*}
%     Donde \(<Var_{i}>\) indica las variables del factor.
%     
%   \item Valores: 
%     \begin{align*}
%       [&[(Factor_1)_{val1}, (Factor_1)_{val2}, . . . , (Factor_1)_{valk}],\\
%        &[(Factor_2)_{val1}, (Factor_2)_{val2}, . . . , (Factor_2)_{valk}],\\
%        & . . . \\
%        &[(Factor_j)_{val1}, (Factor_j)_{val2}, . . . , (Factor_j)_{valk}]]
%     \end{align*}
%     Donde \((Factor_i)_{valn}\) indica el valor numérico correspondiente a cada renglón del \(Factor_i\). Deben tener en cuenta que se toma la última variable especificada en el renglón de Factores como la que cambia más rápidamente.
% \end{itemize}
% 
% 
% \noindent Por ejemplo, el archivo:
% 
% \begin{minted}[mathescape,
%                linenos,
%                numbersep=5pt,
%                gobble=2,
%                frame=lines,
%                fontsize=\scriptsize,
%                framesep=2mm]{text}
%   Variables: {A : [0, 1], B : [0, 1], C : [0, 1, 2]}
%   Factores: [[A, B], [A, B, C]]
%   Valores: [[0.1, 0.9, 0.5, 0.1], [0.25, 0.15, 0.35, 0.1, 0.65, 0.8, 0.6, 0.8, .1, .2, .4, .25]]
% \end{minted}
% 
% \noindent representa los factores:\par
% 
% \begin{figure}[H]
%     \centering
%     \begin{subfigure}[b]{0.4\textwidth}
%         \centering
%         \begin{tabular}{ l  c | r }
%           A & B & $\phi$(A,B)\\ \hline
%           0 & 0 & 0.1  \\ \hline
%           0 & 1 & 0.9  \\ \hline
%           1 & 0 & 0.5  \\ \hline
%           1 & 1 & 0.1  \\
%         \end{tabular}
%         \caption{Factor 1}
%     \end{subfigure}
%     ~ 
%     \begin{subfigure}[b]{0.4\textwidth}
%         \centering
%         \begin{tabular}{ l l  c | r }
%           A & B & C &  \(\phi(A,B,C)\)\\ \hline
%           0 & 0 & 0 & 0.25  \\ \hline
%           0 & 0 & 1 & 0.15  \\ \hline
%           0 & 0 & 2 & 0.35  \\ \hline
%           0 & 1 & 0 & 0.1  \\ \hline
%           0 & 1 & 1 & 0.65  \\ \hline
%           0 & 1 & 2 & 0.8 \\ \hline
%           1 & 0 & 0 & 0.6  \\ \hline
%           1 & 0 & 1 & 0.8  \\ \hline
%           1 & 0 & 2 & 0.1  \\ \hline
%           1 & 1 & 0 & 0.2  \\ \hline
%           1 & 1 & 1 & 0.4  \\ \hline
%           1 & 1 & 2 & 0.25 \\
%         \end{tabular}
%         \caption{Factor 2}
%     \end{subfigure}
%     \caption{Factores representados en el archivo}
% \end{figure}


\noindent Esta práctica deberá ser implementada usando \classname{Python}.

\section{Requisitos y resultados}

Deberán hacer casos de prueba para marginalización, reducción, normalización y multiplicación de factores. Basta con incluir un pequeño \textit{script} con sus casos de prueba. No es necesario que lo hagan a prueba de todo, pero sí que funcionen cuando se cumplen los prerrequisitos para cada operación.

Para evaluar y calificar la práctica es necesario que se implementen todos los métodos mencionados e indicados, respetando las especificaciones de estilo y documentación del lenguaje de programación que usarán.
Es completamente válido utilizar bibliotecas adicionales si lo consideran necesario, así como la creación y uso de sus propios métodos auxiliares si lo desean.

En la figura~\ref{fig:P5solution} pueden apreciar el resultado de crear 3 factores y ejecutar las operaciones de multiplicación, normalización, reducción y marginalización entre ellos. 

\begin{figure}[H]
  \centering
  \begin{subfigure}[b]{.45\textwidth}
    \includegraphics[scale=0.28]{factores/screen1.png}
    \caption{Creación de factores f1 y f2}
  \end{subfigure}
  ~
  \begin{subfigure}[b]{.45\textwidth}
    \includegraphics[scale=0.28]{factores/screen2.png}
    \caption{Multiplicando f1 y f2, para obtener f3}
  \end{subfigure}


  \begin{subfigure}[b]{.45\textwidth}
    \includegraphics[scale=0.28]{factores/screen3.png}
    \caption{Normalizando f3}
  \end{subfigure}
  ~
  \begin{subfigure}[b]{.45\textwidth}
    \includegraphics[scale=0.28]{factores/screen4.png}
    \caption{Reduciendo y marginalizando}
  \end{subfigure}
  \caption{Ejecución del código solución: crea dos factores, los multiplica, normaliza, reduce y marginaliza}
  \label{fig:P5solution}
\end{figure}

No olviden documentar y comentar su código.
