%----------------------------------------------------------------------------------------
% STRIPS
%----------------------------------------------------------------------------------------

En este proyecto implementarás una pequeña máquina de inferencias utilizando los conocimientos de la primera parte del curso.

\section{Meta}

El alumno experimentará de primera mano cómo programar los elementos esenciales de una máquina de inferencias para un sistema experto.

\section{Objetivos}

\begin{compactitem}
 \item Que el alumno se familiarice con el uso del lenguaje de descripción PDDL para expresar
       los pormenores de un problema de planeación.
 \item Familiarizarse con el uso del diseño orientado a objetos en el código, para facilitar
       la implementación de los algoritmos que resolverán el problema de planeación.
 \item Que el alumno implemente un algoritmo de búsqueda que trabaje sobre objetos en un
       dominio PDDL.
 \item Que el alumno valore la relevancia del algoritmo de unificación, en un caso sencillo,
       para la identificación de acciones aplicables a un estado del dominio.
\end{compactitem}


\begin{auxcode}
 \caption{Máquina de inferencias}
 \centering
 \hurl{\auxprefix ia-maquina-de-inferencias}
\end{auxcode}

\section{Desarrollo}

En este paquete se te entrega:
\begin{itemize}
 \item Una copia de la gramática BNF del lenguaje para descripción de problemas de planeación PDDL.
       No utilizarás todas las opciones que se incluyen, así que sólo revísala para que te des una 
       idea de lo que se puede hacer.
 \item Un script de \Python\ con la definición de las clases que representarán a los objetos en
       el dominio, hechos, acciones posibles, estado inicial y meta de un problema de planeación.
       Lee con cuidado la documentación.  Estas clases corresponden a los elementos de PDDL que
       vas a utilizar.  Observa que las clases listan los atributos, pero esta representación es
       independiente de la notación con paréntesis de PDDL.
 %\item En las notas del curso sobre STRIPS se reproduce un ejemplo tomado de \cite{Ghallab2004}.
\end{itemize}


\subsection{Escenario de prueba}

El programa que vas a realizar deberá poder resolver un pequeño problema de planeación, definido según la notación de PDDL.  El dominio siguiente se puede utilizar para probarlo.

\begin{lstlisting}[language=Lisp]
;; Especificación en PDDL1 de un dominio DWR simplificado
 
(define (domain platform-worker-robot)
  (:requirements :strips :typing)
  (:types
    contenedor
    pila	    ;tiene una base y una pila de contenedores.
    grúa	    ;sostiene máximo un contenedor.
  )
  (:predicates
    (sostiene ?k - grúa ?c - contenedor) ;la grúa ?k sostiene al contenedor ?c
    (libre ?k - grúa)                    ;la grúa ?k está libre
    (en ?c - contenedor ?p - pila)       ;el contenedor ?c está en la pila ?p
    (hasta_arriba ?c - contenedor ?p - pila) ;el contenedor ?c se encuentra hasta arriba de la pila ?p
    (sobre ?k1 - contenedor ?k2 - contenedor);el contenedor ?k1 está sobre el contenedor ?k2
  )
  ;; toma un contenedor de una pila con la grúa
  (:action toma
     :parameters (?k - grúa ?c - contenedor ?p - pila)
     :vars (?otro - contenedor)           ;variables locales (extra a PDDL)
     :precondition (and (libre ?k) (en ?c ?p) 
                       (hasta_arriba ?c ?p) (sobre ?c ?otro))
     :effect (and (sostiene ?k ?c) (hasta_arriba ?otro ?p)
                (not (en ?c ?p)) (not (hasta_arriba ?c ?p))
                (not (sobre ?c ?otro)) (not (libre ?k))))

  ;; pone al contenedor sostenido por la grúa en una pila
  (:action pon                                 
     :parameters (?k - grúa ?c - contenedor ?p - pila)
     :vars (?otro - contenedor)           ;variables locales
     :precondition (and (sostiene ?k ?c) (hasta_arriba ?otro ?p))
     :effect (and (en ?c ?p) (hasta_arriba ?c ?p) (sobre ?c ?otro)
                 (not (hasta_arriba ?otro ?p)) (not (sostiene ?k ?c))
                 (libre ?k))))
)
\end{lstlisting}

A continuación se incluye un ejemplar de problema.  Puedes diseñar otros tú y verificar que tu programa también los resuelva correctamente.

\begin{lstlisting}[language=Lisp]
;; un problema sencillo del dominio DWR simplificado
(define (problem dwrpb1)
  (:domain platform-worker-robot)

  (:objects
   k1 k2 - grúa
   p1 q1 p2 q2 - pila
   ca cb cc cd ce cf pallet - contenedor)

  (:init
   (en ca p1)
   (en cb p1)
   (en cc p1)

   (en cd q1)
   (en ce q1)
   (en cf q1)

   (sobre ca pallet)
   (sobre cb ca)
   (sobre cc cb)

   (sobre cd pallet)
   (sobre ce cd)
   (sobre cf ce)

   (hasta_arriba cc p1)
   (hasta_arriba cf q1)
   (hasta_arriba pallet p2)
   (hasta_arriba pallet q2)

   (libre k1)
   (libre k2))

   ;; el problema consiste en mover todos contenedores
   ;; ca y cc a la pila p2, los demás a q2
   (:goal
     (and (en ca p2)
          (en cb q2)
          (en cc p2)
          (en cd q2)
          (en ce q2)
          (en cf q2))
    )
)
\end{lstlisting}

\subsection{Tareas a realizar}

Las tareas que debes realizar son las siguientes:
\begin{enumerate}
\subsubsection{Parte I}
 \item Dibuja el estado inicial.  Puedes incluirlo como comentario en tu código como arte ASCII o adjuntar un archivo de imagen (no debe ser un archivo pesado).

 \item Crea a mano los objetos correspondientes al dominio y problema anteriores.
       Imprímelos y observa que todo esté correcto.

 \item Observa que el estado del mundo se representa como una lista de predictados aterrizados.
       Agrega el código necesario (clases o funciones) para que, dado un estado determine si una
       acción es aplicable o no y con qué sustitución (qué objeto se asigna a qué variable).

 \item Crea una función para aplicar una función aplicable sobre el estado actual y devolver el estado sucesor.  Esta función debe clonar el estado y transformar al clon para que describa ahora al estado sucesor.

 \textbf{TIP:}  Recuerda que al clonar, los objetos nuevos son distintos a sus originales y sus códigos hash cambian, define tu propia función para que puedas determinar si dos predicados son iguales o no, considerando que no son el mismo objeto, sólo tienen valores equivalentes.

 \item Agrega el código para determinar si un estado satisface las condiciones indicadas en el campo \emph{meta}.

 \item Incluye código con pruebas que muestren que tus implementaciones son correctas.
       
\subsubsection{Parte II}
 \item Ahora agrega el código necesario para realizar una búsqueda en amplitud de la secuencia
       de acciones que se debe seguir para alcanzar la meta.  Haz que imprima lo que está
       haciendo de modo que puedas ver si se comporta como deseas.
 \item Utiliza tu programa para buscar una solución al problema anterior.
       ¿Es suficiente información? ¿Cómo se comporta tu programa?
\end{enumerate}
